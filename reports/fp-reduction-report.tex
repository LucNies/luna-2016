% Template for CAD in medical imaging - Project Report
%          spconf.sty  - ICASSP/ICIP LaTeX style file, and
%          IEEEbib.bst - IEEE bibliography style file.
% --------------------------------------------------------------------------
\documentclass{article}
\usepackage{spconf,amsmath,graphicx}
\usepackage{subfigure}

% Title
\title{Computer Aided Diagnosis\\LUNA16: False Positive Reduction}
%
\makeatletter
\def\@name{ \emph{Luc Nies (s4136748)}, \emph{Tom van de Poll (s4106512)}, \emph{Harmen Prins (s4132297)}, \\ \emph{Steven Reitsma (s4132343)} \& \emph{Inez Wijnands (s4149696)}}
\makeatother

\address{Radboud University Nijmegen}
%
% For example:
% ------------
%\address{School\\
%	Department\\
%	Address}
%
% Two addresses (uncomment and modify for two-address case).
% ----------------------------------------------------------
%\twoauthors
%  {A. Author-one, B. Author-two\sthanks{Thanks to XYZ agency for funding.}}
%	{School A-B\\
%	Department A-B\\
%	Address A-B}
%  {C. Author-three, D. Author-four\sthanks{The fourth author performed the work
%	while at ...}}
%	{School C-D\\
%	Department C-D\\
%	Address C-D}
%
% More than two addresses
% -----------------------
% \name{Author Name$^{\star \dagger}$ \qquad Author Name$^{\star}$ \qquad Author Name$^{\dagger}$}
%
% \address{$^{\star}$ Affiliation Number One \\
%     $^{\dagger}$}Affiliation Number Two
%
\begin{document}
%\ninept
%
\maketitle
%


% Introduction
\section{Introduction}
\label{sec:intro}
\emph{LUng Nodule Analysis 2016 (LUNA16)} challenge

% Method
\section{Method}\label{sec:method}

\subsection{Fully convolutional networks}
\subsubsection{Lung segmentation}


\subsubsection{Nodule detection}
\label{sec:fcn}


\subsubsection{False positive reduction}

\subsubsection{Analysing results with FROC}


% Experiments
%\section{Experiments}\label{sec:experiments}

% Results
\section{Results}\label{sec:results}


\appendix
\section{Contributions}
\textbf{Luc Nies:} \\
\\
\textbf{Tom van de Poll:} \\
\\
\textbf{Harmen Prins:} \\
\\
\textbf{Steven Reitsma:} \\
\\
\textbf{Inez Wijnands:} 

% References should be produced using the bibtex program from suitable
% BiBTeX files (here: strings, refs, manuals). The IEEEbib.bst bibliography
% style file from IEEE produces unsorted bibliography list.
% -------------------------------------------------------------------------
\bibliographystyle{IEEEbib}

\begin{thebibliography}{}
\bibitem{long}
J. Long, E. Shelhamer \& T. Darrell (2015). Fully convolutional networks for semantic segmentation. \emph{Proceedings of the IEEE Conference on Computer Vision and Pattern Recognition}: 3431--3440.

\bibitem{simonyan}
K. Simonyan \& A. Zisserman (2014). Very deep convolutional networks for large-scale image recognition. \emph{arXiv:1409.1556}

%\bibitem{leemput}
%S. van de Leemput, F. Dorssers \& B.E. Bejnordi (2015). A novel spherical shell filter for reducing false positives in automatic detection of pulmonary nodules in thoracic CT scans. \emph{Proceedings SPIE Medical Imaging, 9414}:94142P.

\end{thebibliography}



\end{document}
