% Template for CAD in medical imaging - Project Report
%          spconf.sty  - ICASSP/ICIP LaTeX style file, and
%          IEEEbib.bst - IEEE bibliography style file.
% --------------------------------------------------------------------------
\documentclass{article}
\usepackage{spconf,amsmath,graphicx}
\usepackage{subfigure}

% Title
\title{Computer Aided Diagnosis\\LUNA16: False Positive Reduction}
%
\makeatletter
\def\@name{ \emph{Luc Nies (s4136748)}, \emph{Tom van de Poll (s4106512)}, \emph{Harmen Prins (s4132297)}, \\ \emph{Steven Reitsma (s4132343)} \& \emph{Inez Wijnands (s4149696)}}
\makeatother

\address{Radboud University Nijmegen}
%
% For example:
% ------------
%\address{School\\
%	Department\\
%	Address}
%
% Two addresses (uncomment and modify for two-address case).
% ----------------------------------------------------------
%\twoauthors
%  {A. Author-one, B. Author-two\sthanks{Thanks to XYZ agency for funding.}}
%	{School A-B\\
%	Department A-B\\
%	Address A-B}
%  {C. Author-three, D. Author-four\sthanks{The fourth author performed the work
%	while at ...}}
%	{School C-D\\
%	Department C-D\\
%	Address C-D}
%
% More than two addresses
% -----------------------
% \name{Author Name$^{\star \dagger}$ \qquad Author Name$^{\star}$ \qquad Author Name$^{\dagger}$}
%
% \address{$^{\star}$ Affiliation Number One \\
%     $^{\dagger}$}Affiliation Number Two
%
\begin{document}
%\ninept
%
\maketitle
%


% Introduction
\section{Introduction}
\label{sec:intro}
For the \emph{LUng Nodule Analysis 2016 (LUNA16)} challenge, we want to detect pulmonary nodules in low-dose thoracic CT images as accurately as possible. Initially the challenge is separated into three different phases; lung segmentation, candidate detection and false positive reduction. For the first two phases, we tried two different approaches. Firstly, we used deep learning to create a network that could train on the images and could recognize areas of interest. For the first phase these areas were the lung regions, for the second these areas were possible nodules in the lung regions. This way, we could treat both phases as a segmentation problem and could use the same network. Secondly, we used a more traditional approach using image processing techniques like region growing and blob detection in the images to complete these phases.\\
\indent After the second phase, we had so far obtained marginally better results for the deep learning approach than the traditional approach. Therefore, we focus solely on our deep learning approach for the third phase.\\
\\
\indent As mentioned, the third phase of our lung nodule detection challenge is false positive reduction. Heretofore, we have focused on identifying as many possible candidates as possible to include all annotated nodules in our selection. Thus, we only aimed at a high recall, but did not care much for precision. In this phase, we aim to obtain both high recall and high precision. We approached this problem by going back to the initial candidate selection, instead of taking our candidates and making a subselection to reduce the false positives. By tuning our network parameters and trying different settings overall, we tried to achieve better initial candidate selection and thus not needing false positive reduction, merging the second and third phase of the challenge. \\
\indent Our approach and its results are explained in further detail in the following sections.


% Method
\section{Method}\label{sec:method} 
\subsection{Fully convolutional networks}
\subsubsection{Lung segmentation post-processing}
We've made use of the segmentations created by the network in earlier phases of the project, in order to more carefully select training patches from the images for training. A small issue with the segmentations created by the network is that there is some noise surrounding the actual lung segmentation. In order to use these segmentations for patch selection, the noise has to be removed from the images. In order to remove the noise, we perform an connected component analysis with a very small 3d-6 neighborhood. This divides the noise next to the segmentations into a large number of small connected components, but still sees the lungs and one large connected component. After this step, only the largest component in the image is selected and used as the real segmentation. This method works for every image in the dataset.
\cite{long}

\subsubsection{Nodule detection}
\label{sec:fcn}


\subsubsection{False positive reduction}

\subsubsection{Analysing results with FROC}


% Experiments
%\section{Experiments}\label{sec:experiments}

% Results
\section{Results}\label{sec:results}


\appendix
\section{Contributions}
\textbf{Luc Nies:} \\
\\
\textbf{Tom van de Poll:} \\
\\
\textbf{Harmen Prins:} \\
\\
\textbf{Steven Reitsma:} \\
\\
\textbf{Inez Wijnands:} 

% References should be produced using the bibtex program from suitable
% BiBTeX files (here: strings, refs, manuals). The IEEEbib.bst bibliography
% style file from IEEE produces unsorted bibliography list.
% -------------------------------------------------------------------------
\bibliographystyle{IEEEbib}

\begin{thebibliography}{}
\bibitem{long}
J. Long, E. Shelhamer \& T. Darrell (2015). Fully convolutional networks for semantic segmentation. \emph{Proceedings of the IEEE Conference on Computer Vision and Pattern Recognition}: 3431--3440.

\bibitem{simonyan}
K. Simonyan \& A. Zisserman (2014). Very deep convolutional networks for large-scale image recognition. \emph{arXiv:1409.1556}

%\bibitem{leemput}
%S. van de Leemput, F. Dorssers \& B.E. Bejnordi (2015). A novel spherical shell filter for reducing false positives in automatic detection of pulmonary nodules in thoracic CT scans. \emph{Proceedings SPIE Medical Imaging, 9414}:94142P.

\end{thebibliography}



\end{document}
